\documentclass[12pt, a4paper]{article}
\usepackage[a4paper, includeheadfoot, mag=1000, left=2cm, right=1.5cm, top=1.5cm, bottom=1.5cm, headsep=0.8cm, footskip=0.8cm]{geometry}
% Fonts
\usepackage{fontspec, unicode-math}
\setmainfont[Ligatures=TeX]{CMU Serif}
\setmonofont{CMU Typewriter Text}
\usepackage[english, russian]{babel}
% Indent first paragraph
\usepackage{indentfirst}
\setlength{\parskip}{5pt}
% Diagrams
\usepackage{graphicx}
\usepackage{float}
% Page headings
\usepackage{fancyhdr}
\pagestyle{fancy}
\renewcommand{\headrulewidth}{0pt}
\setlength{\headheight}{16pt}
%\newfontfamily\namefont[Scale=1.2]{Gloria Hallelujah}
\fancyhead{}
\begin{document}

% Title page
\include{spectitlepage}
\section{Описание команд}
\subsection{Команды для повторения}
\begin{itemize}
  % LS
  \item \textbf{\texttt{ls}} -- (eng. \textit{list}) -- Показывает содержимое
    директории. \texttt{ls [OPTION]... [FILE]...} - по умолчанию FILE = ., т.е.
    ls по умолчанию показывает содержимое текущей директории.
    \begin{itemize}
      \item \textbf{\texttt{-a, --all}} -- отображать скрытые файлы (имя
        начинается с .)
      \item \textbf{\texttt{-A, --almost-all}} -- отображать скрытые файлы,
        кроме . и ..
      \item \textbf{\texttt{-d, --directory}} -- отображать информацию только
        о текущей директории, но не о её содержимом
      \item \textbf{\texttt{-h, --human-readable}} -- отображать информацию о
        размере файлов с использованием приставок (K, M, G etc.)
      \item \textbf{\texttt{--si}} -- аналог -h, но используются степени 10, а
        не 2
      \item \textbf{\texttt{-i, --inode}} - отображать index number каждого файла
      \item \textbf{\texttt{-l}} - расширенный формат отображения
    \end{itemize}
  % PWD
  \item \textbf{\texttt{pwd}} -- (eng. \textit{present working directory}) --
    отобразить имя текущего рабочего каталога. \texttt{pwd [OPTION]}
    \begin{itemize}
      \item \textbf{\texttt{-P, --physical}} -- не учитывать символьные ссылки
      \item \textbf{\texttt{-L, --logical}} -- использовать переменную окружения
        \texttt{PWD}, даже с символьными ссылками
    \end{itemize}
  % CD
  \item \textbf{\texttt{cd}} -- (eng. \textit{change directory}) -- изменить
    текущий рабочий каталог. \texttt{cd [OPTION]... [dir]}. cd с неуказанным
    dir изменит текущий рабочий каталог на содержащийся в переменной окружения
    \texttt{HOME}. Если указать - как dir, то будет осуществлен переход по
    переменной окружения \texttt{\$OLDPWD}. После успешного перехода \texttt{PWD}
    становится \texttt{\$OLDPWD}
    \begin{itemize}
      \item \textbf{\texttt{-P, --physical}} -- не учитывать символьные ссылки
      \item \textbf{\texttt{-L, --logical}} -- учитывать символьные ссылки
    \end{itemize}
  % RM
  \item \textbf{\texttt{rm}} -- (eng. \textit{remove}) -- удалить файлы.
    \texttt{rm [OPTION]... [FILE]}.
    \begin{itemize}
      \item \textbf{\texttt{-f, --force}} -- игнорировать немуществующие файлы
        и опции, никогда не запрашивать подтверждения на удаление
      \item \textbf{\texttt{-i}} -- запрашивать подтверждение на каждое удаление
      \item \textbf{\texttt{-I}} -- запросить подтверждение на удаление в случае
        удаления больше трех файлов или рекурсивного удаление сожержимого
        каталога
      \item \textbf{\texttt{-r, -R, --recursive}} -- удалить все каталоги и их
        содержимое
      \item \textbf{\texttt{-d, -dir}} -- удалить пустые каталоги
    \end{itemize}
  % MV
  \item \textbf{\texttt{mv}} -- (eng. \textit{move}) -- переместить/переименовать
    файлы. \texttt{mv [OPTION]... SOURCE DEST}
    \begin{itemize}
      \item \textbf{\texttt{-f, --force}} -- не запрашивать подтверждение на
        перезапись
      \item \textbf{\texttt{-i, --interactive}} -- запрашивать подтверждение на
        перезапись
      \item \textbf{\texttt{-n, --no-clobber}} -- не перезаписывать
      \item \textbf{\texttt{-u, --update}} -- исполнить перемещение только если
        SOURCE новее DEST или DEST не найден
    \end{itemize}
  % CP
  \item \textbf{\texttt{cp}} -- (end. \textit{copy}) -- копировать файлы и каталоги.
    \texttt{cp [OPTION]... SOURCE DEST}
    \begin{itemize}
      \item \textbf{\texttt{-f, --force}} -- если не удается открыть DEST,
        удалить его и попробовать снова
      \item \textbf{\texttt{-i, --interactive}} -- запрашивать подтверждение на
        перезапись
      \item \textbf{\texttt{-n, --no-clobber}} -- не перезаписывать
      \item \textbf{\texttt{-u, --update}} -- исполнить копирование только если
        SOURCE новее DEST или DEST не найден
      \item \textbf{\texttt{-l, --link}} -- создавать hard link вместо копирования
      \item \textbf{\texttt{-s, --symbolic-link}} -- создавать soft link вместо
        копирования
      \item \textbf{\texttt{-r, -R, --recursive}} -- скопировать все каталоги и
        их содержимое
    \end{itemize}
  \item \textbf{\texttt{mkdir}} -- (eng. \textit{make directory}) -- создать
    директорию. \texttt{mkdir [OPTION]... DIRECTORY...}
    \begin{itemize}
      \item \textbf{\texttt{-m, --mode}} -- настроить права доступа (см. chmod)
      \item \textbf{\texttt{-p, --parents}} -- не показывать ошибку, если
        DIRECTORY уже существует, создать дерево директорий если требуется
      \item \textbf{\texttt{-v, --verbose}} -- показывать информационное
        сообщение о создании каждой директории
    \end{itemize}
  % RMDIR
  \item \textbf{\texttt{rmdir}} -- (eng. \textit{remove directory}) -- удалить
    пустую директорию. \texttt{rmdir [OPTION]... DIRECTORY}
  \begin{itemize}
    \item \textbf{\texttt{-p, --parents}} -- удалить потомков
    \item \textbf{\texttt{-v, --verbose}} -- показывать информационное
      сообщение для каждой удаляемой директории
    \item \textbf{\texttt{--ignore-fail-on-non-empty}} -- не показывать
      сообщение об ошибке, если директория не пустая
  \end{itemize}
  % TYPE
  \item \textbf{\texttt{type}} -- (eng. \textit{type}) -- определяет, как
    определенное выражение будет интерпретироваться командной строкой.
    \texttt{type [OPTION]... [NAME]...}
  \begin{itemize}
    \item \textbf{\texttt{-a}} -- показывать все расположения исполняемых
      файлов по имени NAME
    \item \textbf{\texttt{-t}} -- упрощенный вывод
    \item \textbf{\texttt{-P}} -- искать в \texttt{PATH}
  \end{itemize}
  % FILE
  \item \textbf{\texttt{file}} -- (eng. \textit{file}) -- определить тип файла.
    \texttt{file [OPTION]... file}
  \begin{itemize}
    \item \textbf{\texttt{-b, --brief}} -- не отображать имя файла в выводе
    \item \textbf{\texttt{-E}} -- при ошибке фаловой системы показать сообщение
      об ошибке и остановить выполнение
    \item \textbf{\texttt{-e, --exclude}} -- не проверять файл на соответствие
      типу, указанному в \texttt{testname}
  \end{itemize}
  % FIND
  \item \textbf{\texttt{find}} -- (eng. \textit{find}) -- поиск файлов,
  удовлетворяющих паттерну, в иерархии директорий. \texttt{find [OPTION]...
  [starting-point] [expression]}
  \begin{itemize}
    \item \textbf{\texttt{-name}} -- искать файлы по имени
    \item \textbf{\texttt{-type}} -- искать файлы определенного типа (\texttt{
      f, d, l etc.})
    \item \textbf{\texttt{-user}} -- искать файлы, принадлежащие определенному
     пользователю
    \item \textbf{\texttt{-mtime}} -- искать файлы, дата изменения которых не
     превышает указанный срок
    \item \textbf{\texttt{-iname}} -- искать файлы по имени, игнорируя регистр
    \item \textbf{\texttt{-not}} -- вывести файлы, которые не удовлетворяют
      данным параметрам
    \item \textbf{\texttt{-maxdepth}} -- искать файлы в иерархии директорий не
      глубже заданного параметра
  \end{itemize}
  % CHMOD
  \item \textbf{\texttt{chmod}} -- (eng. \textit{change mode}) -- модифицировать
    права доступа файлов и каталогов. \texttt{chmod [OPTION]... [MODE] FILE}
  \begin{itemize}
    \item \textbf{\texttt{-v, --verbose}} -- показать информационное сообщение
      для каждого обработанного файла
    \item \textbf{\texttt{-c, --changes}} -- показать информационное сообщение
      только если модификация произошла
    \item \textbf{\texttt{-f, --silent, --quiet}} -- не показывать сообщения об
      ошибках
    \item \textbf{\texttt{--reference}} -- выставить права доступа аналогично
      файлу, указанному как параметр
    \item \textbf{\texttt{-R, --recursive}} -- выставить права на все файлы в
      директории рекурсивно
  \end{itemize}
  % LN
  \item \textbf{\texttt{ln}} -- (eng. \textit{link}) -- создать ссылку на файл
    или директорию. \texttt{ln [OPTION] TARGET LINK\_NAME}
  \begin{itemize}
    \item \textbf{\texttt{-f, --force}} -- перезаписать файлы, если LINK\_NAME
      существует
    \item \textbf{\texttt{-i, --interactive}} -- запрашивать подтверждение на
      перезапись
    \item \textbf{\texttt{-s, --symbolic}} -- создать soft link
    \item \textbf{\texttt{-P, --physical}} -- создавать hard link прямо на
      soft link
    \item \textbf{\texttt{-L, --logical}} -- разворачивать soft link и создавать
      hard link на то, на что она указывает
    \item \textbf{\texttt{-v, --verbose}} -- показывать информационное сообщение
      при создании каждой ссылки
  \end{itemize}
  % WC
  \item \textbf{\texttt{wc}} -- (eng. \textit{word count}) -- вывести информацию
    о файле -  число переводов строк, слов и байт для каждого указанного файла
    и итоговую строку, если было задано несколько файлов. Если входной файл не
    задан, или равен ‘-‘, то данные считываются со стандартного ввода.
    \texttt{wc [OPTION]... [FILE]...}
    \begin{itemize}
      \item \textbf{\texttt{-c, --bytes}} -- вывести количество байт
      \item \textbf{\texttt{-m, --chars}} -- вывести количество символов
      \item \textbf{\texttt{-l, --lines}} -- вывести количество переводов строк
      \item \textbf{\texttt{-w, --words}} -- вывести количество слов
    \end{itemize}
  % TEE
  \item \textbf{\texttt{tee}} -- (named after the \texttt{T-splitter} used in
    plumbing) -- чтение из стандартного потока ввода и запись в стандартный
    поток вывода или файл. \texttt{tee [OPTION]... [FILE]...}
  \begin{itemize}
    \item \textbf{\texttt{-a, --append}} -- не перезаписывать файл, дописывать
      в конец
    \item \textbf{\texttt{-i, --ignore-interrupts}} -- игнорировать сигналы
      прерывания
  \end{itemize}
  % CAT
  \item \textbf{\texttt{cat}} -- (eng. \textit{concatenate}) -- объединить
    последовательно указанные файлы и вывести их содержимое в стандартный
    поток вывода. \texttt{cat [OPTIONS]... [FILE]...}
  \begin{itemize}
    \item \textbf{\texttt{-n, --number}} -- пронумеровать все строчки
    \item \textbf{\texttt{-b, --number-nonblank}} -- пронумеровать все
      непустые строчки
    \item \textbf{\texttt{-v, --show-nonprinting}} -- отображать непечатные
      символы
    \item \textbf{\texttt{-T, --show-tabs}} -- отображать табы как 
      \textasciicircum I
    \item \textbf{\texttt{-E, --show-ends}} -- отображать \$ на конце строчек
    \item \textbf{\texttt{-t}} -- эквивалентно \texttt{-vT}
    \item \textbf{\texttt{-e}} -- эквивалентно \texttt{-vE}
    \item \textbf{\texttt{-A, --show-all}} -- эквивалентно \texttt{-vET}
  \end{itemize}
  % TAIL
  \item \textbf{\texttt{tail}} -- (eng. \textit{tail}) -- вывести последние
    строчки файла. \texttt{tail [OPTION]... [FILE]...}
  \begin{itemize}
    \item \textbf{\texttt{-c, --bytes}} -- вывести определенное количество байт
      файла с конца
    \item \textbf{\texttt{-n, --lines}} -- вывести определенное количество
      строк файла с конца (по умолчанию 10)
    \item \textbf{\texttt{-q, --silent, --quiet}} -- не печатать заголовки
      файлов
    \item \textbf{\texttt{-v, --verbose}} -- всегда печатать заголовки файлов
    \item \textbf{\texttt{-z, --zero-terminated}} -- считать NUL за символ
      новой строки
    \item \textbf{\texttt{--retry}} -- повторно пытаться открыть файл, если он
      недоступен
    \item \textbf{\texttt{-f, --follow}} -- отображать изменения файла по мере
      его наполнения
    \item \textbf{\texttt{--pid}} -- завершить работу с завершением указанного
      процесса
  \end{itemize}
  % HEAD
  \item \textbf{\texttt{head}} -- (eng. \textit{head}) -- вывести первые строчки
    файла. \texttt{head [OPTION]... [FILE]...}
  \begin{itemize}
    \item \textbf{\texttt{-c, --bytes}} -- вывести определенное количество байт
      файла с начала
    \item \textbf{\texttt{-n, --lines}} -- вывести определенное количество
      строк файла с начала (по умолчанию 10)
    \item \textbf{\texttt{-q, --silent, --quiet}} -- не печатать заголовки
      файлов
    \item \textbf{\texttt{-v, --verbose}} -- всегда печатать заголовки файлов
    \item \textbf{\texttt{-z, --zero-terminated}} -- считать NUL за символ
      новой строки
  \end{itemize}
  % MORE
  \item \textbf{\texttt{more}} -- (eng. \textit{more}) -- команда для просмотра
    файлов в страничном режиме. \texttt{more [OPTIONS]... [FILE]...}
  \begin{itemize}
    \item \textbf{\texttt{-d}} -- иное поведение при использовании недопустимых
      клавиш
    \item \textbf{\texttt{-l}} -- не воспринимать \textasciicircum L как
      специальный символ
    \item \textbf{\texttt{-s}} -- объединять последовательности пустых строчек
      в одну
    \item \textbf{\texttt{-num}} -- размер экрана в строках
    \item \textbf{\texttt{+num}} -- номер начальной строки
    \item \textbf{\texttt{+/}} -- паттерн для поиска по документу перед его
      открытием, в случае ненахождения перед открытием будет выведено
      соответствующее сообщение
  \end{itemize}
  % PG
  \item \textbf{\texttt{pg}} -- (eng. \textit{page}) -- команда для просмотра
    файлов в страничном режиме \texttt{pg [OPTION]... [FILE]...}
  \begin{itemize}
    \item \textbf{\texttt{-num}} -- размер экрана в строчках
    \item \textbf{\texttt{-e}} -- выводить (EOF) на конце файла
    \item \textbf{\texttt{-c}} -- очистить экран перед выводом новой страницы
    \item \textbf{\texttt{+num}} -- номер начальной строки
    \item \textbf{\texttt{+/pattern/}} -- начать со строки, в которой найден
      паттерн
  \end{itemize}
  \item \textbf{\texttt{touch}} -- (eng. \textit{touch}) -- обновление времени
    доступа к файлу и времени его изменения. \texttt{touch [OPTION]... FILE...}
    По умолчанию создает новый файл, если FILE отсутствует
    \begin{itemize}
      \item \textbf{\texttt{-a}} -- обновить только время доступа
      \item \textbf{\texttt{-m}} -- обновить только время модификации
      \item \textbf{\texttt{-d, --date}} -- спарсить строку и использовать
        вместо текущего времени
      \item \textbf{\texttt{-r, --reference}} -- выставить время согласно
        параметрам другого файла
      \item \textbf{\texttt{-t}} -- использовать данный timestamp вместо
        текущего времени
      \item \textbf{\texttt{--time}} -- обновить только определенное время
        (acces, atime, mtime etc.)
      \item \textbf{\texttt{-c, --no-create}} -- не создавать новый файл если
        FILE отсутствует
    \end{itemize}
\end{itemize}
\end{document}
