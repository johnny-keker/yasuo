\documentclass[12pt, a4paper]{article}
\usepackage[a4paper, includeheadfoot, mag=1000, left=2cm, right=1.5cm, top=1.5cm, bottom=1.5cm, headsep=0.8cm, footskip=0.8cm]{geometry}
% Fonts
\usepackage{fontspec, unicode-math}
\setmainfont[Ligatures=TeX]{CMU Serif}
\setmonofont{CMU Typewriter Text}
\usepackage[english, russian]{babel}
% Indent first paragraph
\usepackage{indentfirst}
\setlength{\parskip}{5pt}
% Diagrams
\usepackage{graphicx}
\usepackage{float}
% Source code
\usepackage{verbatim}
\usepackage{listings}
% Page headings
\usepackage{fancyhdr}
\pagestyle{fancy}
\renewcommand{\headrulewidth}{0pt}
\setlength{\headheight}{16pt}
%\newfontfamily\namefont[Scale=1.2]{Gloria Hallelujah}
\fancyhead{}
\begin{document}

% Title page
\begin{titlepage}
\begin{center}

\textsc{ФГАОУ ВО «Санкт-Петербургский национальный исследовательский университет информационных технологий, механики и оптики»\\[4mm]
Кафедра вычислительной техники}
\vfill
\textbf{Лабораторная работа №6\\[4mm]
Системное программное обеспечение\\[16mm]
}
\begin{flushright}
Саржевский Иван Анатольевич
\\[2mm]Группа P3202
\end{flushright}
\vfill
Санкт-Петербург\\[2mm]
2019 г.

\end{center}
\end{titlepage}

\section*{Первое задание}
\begin{enumerate}
  \item Выведите все номера телефонов:\\
    \verb|awk -F ":" '{print $2}' db|\\
    \verb|-F| задает разделитель, выводим вторую колонку
  \item Выведите номер телефона, принадлежащий сотруднику Dan:\\
    \verb|awk -F ":" '/^Dan/ {print $2}' db|\\
    Аналогично предыдущему, но ищем строки, начинающиеся с подстроки \verb|Dan|
  \item Выведите имя, фамилию и номер телефона сотрудницы Susan:\\
    \verb|awk -F ":" '/^Susan/ {print $1, $2}' db|\\
    Аналогично предыдущим примерам, но выведем первую и вторую колонки
  \item Выведите все фамилии, начинающиеся с буквы D\\
    \verb|awk -F "[ :]" '$2~/^D.*/ {print $2}' db|\\
    Зададим разделитель регулярным выражением, проверим что вторая колонка
    начинается с \verb|D|, если это так выведем её
  \item Выведите все имена, начинающиеся с буквы C или E\\
    \verb|awk '/^[CE].*/ {print $1}' db|\\
    Аналогично предыдущему, проверим, что строка начинается с \verb|C| или
    \verb|E|, выведем первую колонку, если это так
  \item Выведите все имена, состоящие только из четырех букв\\
    \verb|awk '/^.... .*/ {print $1}' db|\\
    Если строка подпадает под регулярное выражение - выведем первую колонку
  \item Выведите имена сотрудников, префикс номера телефона которых 916\\
    \verb|awk '/\(916\)/ {print $1}' db|\\
    Аналогично предыдущему
  \item Выведите денежные вклады сотрудника Mike, предваряя каждую сумму знаком \$\\
    \verb|awk -F ":" '/^Mike/ {printf("$%s, $%s, $%s\n", $3, $4, $5)}' db|\\
    Воспользуемся функцией \verb|printf| чтобы отформатировать строку согласно
    заданию
  \item Выведите инициалы всех сотрудников\\
    \verb|awk '{printf("%s. %s.\n", substr($1, 1, 1), substr($2, 1, 1))}' db|\\
    Воспользуемся функцией \verb|substr| для того, чтобы достать инициалы
  \item Создайте командный файл awk, который:
    \begin{enumerate}
      \item Печатает полные имена и номера телефонов всех сотрудников по фамилии Savage
      \item Печатает денежные вклады сотрудника по имени Chet
      \item Печатает сотрудников, денежные вклады которых в первом месяце составили 250\$
      \item Подсчитывает сумму вкладов за каждый месяц в отдельности и вывести это в виде оформленной таблицы
      \item Подсчитывает средний вклад за каждый месяц и выводит результаты округлённо до второго знака после запятой
      \item В конце вывести текущее время и результат выполнения команды ls
    \end{enumerate}
    \lstinputlisting{task.awk}
\end{enumerate}
\section*{Второе задание}
\begin{enumerate}
  \item \verb|nawk '/west/' datafile| : Выведет строки, содержащие подстроку
    \verb|west|
  \item \verb|nawk '/^north/' datafile| : Выведет строки, начинающиеся с
    \verb|north|
  \item \verb=nawk '/^(no|so)/' datafile= : Выведет строки, начинающиеся с
    \verb|no| или \verb|so|
  \item \verb|nawk '{print $3, $2}' datafile| : Выведет третью и вторую колонку
    строчек через разделитель
  \item \verb|nawk '{print $3 $2}' datafile| : Выведет третью и вторую колонку
    строчек слитно
  \item \verb|nawk '{print $0}' datafile| : Выведет строки без изменений
  \item \verb|nawk '{print "Number of fields: "NF}' datafile| : Выведет количество
    колонок в каждой строке
  \item \verb|nawk '/northeast/{print $3, $2}' datafile| : Выведет 3 и 2 колонки
    строчек, содержащих подстроку \verb|northeast|
  \item \verb|nawk '/E/' datafile| : Выведет все строки, содержащие подстроку \verb|Е|
  \item \verb|nawk '/^[ns]/{print $1}' datafile| : Выведет первую колонку строки,
    если она начинается с \verb|n| или \verb|s|
  \item \verb|nawk '$5 ~ /\.[7-9]+/' datafile| : Выведет строку, если пятая
    колонка содержит подстроку - точка и одно или более число от 7 до 9
  \item \verb|nawk '$2 !~ /E/{print $1, $2}' datafile| : Выведет первую и вторую
    колонку, если вторая колонка не содержит подстроку \verb|E|
  \item \verb|nawk '$3 ~ /^Joel/{print $3 " is a nice guy."}' datafile| :
    Если третья колонка начинается с \verb|Joel|, то выведем ее и припишем
    \verb| is a nice guy.|
  \item \verb|nawk '$8 ~ /[1-9][0-2]$/{print $8}' datafile| : Если восьмая колонка
    кончается на число от 1 до 9 за которым следует число от 0 до 2 - выведем её
  \item \verb|nawk '$4 ~ /Chin$/{print "The price is $" $8 "."}' datafile| :
    Если четвертая колонка кончается на \verb|Chin| - выведем "The price is \$"
    и значение восьмой колонки
  \item \verb|nawk '/TJ/{print $0}' datafile| : Вывести все строки, содержащие
    подстроку \verb|TJ|
\end{enumerate}
\end{document}
