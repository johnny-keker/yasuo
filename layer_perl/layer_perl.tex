\documentclass[12pt, a4paper]{article}
\usepackage[a4paper, includeheadfoot, mag=1000, left=2cm, right=1.5cm, top=1.5cm, bottom=1.5cm, headsep=0.8cm, footskip=0.8cm]{geometry}
% Fonts
\usepackage{fontspec, unicode-math}
\setmainfont[Ligatures=TeX]{CMU Serif}
\setmonofont{CMU Typewriter Text}
\usepackage[english, russian]{babel}
% Indent first paragraph
\usepackage{indentfirst}
\setlength{\parskip}{5pt}
% Diagrams
\usepackage{graphicx}
\usepackage{float}
% Source code
\usepackage{verbatim}
% Page headings
\usepackage{fancyhdr}
\pagestyle{fancy}
\renewcommand{\headrulewidth}{0pt}
\setlength{\headheight}{16pt}
%\newfontfamily\namefont[Scale=1.2]{Gloria Hallelujah}
\fancyhead{}
\begin{document}

% Title page
\begin{titlepage}
\begin{center}

\textsc{ФГАОУ ВО «Санкт-Петербургский национальный исследовательский университет информационных технологий, механики и оптики»\\[4mm]
Кафедра вычислительной техники}
\vfill
\textbf{Самостоятельная работа №2\\[4mm]
Системное программное обеспечение\\[4mm]
Знакомство с Perl\\[16mm]
}
\begin{flushright}
Саржевский Иван Анатольевич
\\[2mm]Группа P3202
\end{flushright}
\vfill
Санкт-Петербург\\[2mm]
2019 г.

\end{center}
\end{titlepage}

\section{Регулярные выражения, var="abc123"}
\begin{itemize}
  \item \verb|$var =~ /./| - \verb|true|
  \item \verb|$var =~ /[A-Z]*^/| - \verb|true|
  \item \verb|$var =~ /(\d)2(\1)/| - \verb|false|
  \item \verb|$var =~ /abc$/| - \verb|false|
  \item \verb|$var =~ /1234?/| - \verb|true|
\end{itemize}
\section{Подстановки, var="abc123abc"}
\begin{itemize}
  \item \verb|$var =~ s/abc/def/;| - \verb|def123abc|
  \item \verb|$var =~ s/(?<=\d)[a-z]+/X/g;| - \verb|abc123X|
  \item \verb|$var =~ s/B/W/i;| - \verb|aWc123abc|
  \item \verb|$var =~ s/(.)\d.*\1/d/;| - \verb|abd|
  \item \verb|$var =~ s/(\d+)/$1*2/e;| - \verb|abc246abc|
\end{itemize}
\section{Шаблоны}
\begin{itemize}
  \item \verb=|/a|bc*/|= - \verb|a| или \verb|b|, за которой следует любое
    количество букв \verb|c|
  \item \verb|/[\d]{1,3}/| - любые цифры, объединенные в группы от одного до
    трех
  \item \verb|/\bc[aou]t\b/| - \verb|cat, cot, cut|
  \item \verb|/(xy+z)\.\1/| - \verb|x|, любое количество \verb|y|, затем \verb|z|,
    затем \verb|.|, затем повторение всего, кроме точки
  \item \verb|/^$/| - пустая строка
\end{itemize}
\section{Свои шаблоны}
\begin{itemize}
  \item не менее пяти маленьких латинских букв, либо буква "а", если она стоит в начале строки
    - \verb=/[a-z]{5,}|^a/=
  \item цифра 1 или слово "one" (в любом регистре) - \verb=/1|one/i=
  \item число, возможно, дробное (с десятичной точкой) - \verb|\d+(\.\d+)?|
  \item любая буква, за которой следует гласная, повторяется еще раз
    (пример: "pop", "fifth", "daddy") - \verb|([A-Za-z])[aioueAIOUE]\1|
  \item хотя бы один \verb|"+" - \verb|\++|
\end{itemize}
\end{document}
