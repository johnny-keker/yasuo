\documentclass[12pt, a4paper]{article}
\usepackage[a4paper, includeheadfoot, mag=1000, left=2cm, right=1.5cm, top=1.5cm, bottom=1.5cm, headsep=0.8cm, footskip=0.8cm]{geometry}
% Fonts
\usepackage{fontspec, unicode-math}
\setmainfont[Ligatures=TeX]{CMU Serif}
\setmonofont{CMU Typewriter Text}
\usepackage[english, russian]{babel}
% Indent first paragraph
\usepackage{indentfirst}
\setlength{\parskip}{5pt}
% Diagrams
\usepackage{graphicx}
\usepackage{float}
% Source code
\usepackage{verbatim}
% Page headings
\usepackage{fancyhdr}
\pagestyle{fancy}
\renewcommand{\headrulewidth}{0pt}
\setlength{\headheight}{16pt}
%\newfontfamily\namefont[Scale=1.2]{Gloria Hallelujah}
\fancyhead{}
\begin{document}

% Title page
\begin{titlepage}
\begin{center}

\textsc{ФГАОУ ВО «Санкт-Петербургский национальный исследовательский университет информационных технологий, механики и оптики»\\[4mm]
Кафедра вычислительной техники}
\vfill
\textbf{Лабораторная работа №3\\[4mm]
Системное программное обеспечение\\[16mm]
}
\begin{flushright}
Саржевский Иван Анатольевич
\\[2mm]Группа P3202
\end{flushright}
\vfill
Санкт-Петербург\\[2mm]
2019 г.

\end{center}
\end{titlepage}


\section{Запуск и перевод команды в фоновый режим}
\begin{itemize}
  \item Запуск в фоновом режиме: \texttt{vim /etc/passwd \&}
  \item Перевод в фоновый режим: \texttt{vim /etc/passwd | killall -STOP vimx | bg \%1}
\end{itemize}

\section{Описание изученных комманд}
\begin{itemize}
  \item \textbf{\texttt{ps}} -- (eng. \textit{process state}) -- утилита для просмотра
    информации о работе текущих процессов в системе.
    \begin{itemize}
      \item \textbf{\texttt{-e}} : отображение всех активных процессов (по
        умолчанию - только для текущей оболочки)
      \item \textbf{\texttt{-x}} : отображение процессов, принадлежащих текущему
        пользователю
      \item \textbf{\texttt{-U}} : отобразить процессы, принадлежащие определенному
        пользователю
      \item \textbf{\texttt{-G}} : отобразать процессы, принадлежащие определенной
        группе
      \item \textbf{\texttt{-t}} : отобразить процессы по tty
      \item \textbf{\texttt{--forest}} : отображать деревья процессов
      \item \textbf{\texttt{-o}} : указать формат вывода
      \item \textbf{\texttt{-C}} : выбрать конкретный процесс по имени (с дочерними)
      \item \textbf{\texttt{--sort}} : выбрать параметр для сортировки
    \end{itemize}
    \texttt{ps -eo cmd,\%mem,\%cpu --sort=\%mem} : вывести самые дорогие
    по используемой памяти процессы
  \item \textbf{\texttt{crontab}} -- утилита для работы с \texttt{cron} - демоном,
    использующимся для периодического выполнения заданий в определенное время
    \begin{itemize}
      \item {\textbf{\texttt{-l}}} : вывести содержание файла расписания для
        текущего пользователя
      \item {\textbf{\texttt{-r}}} : удалить текущий файл расписания
      \item {\textbf{\texttt{-e}}} : редактировать текущий файл расписания
    \end{itemize}
    \texttt{crontab -e\\5 4 * * sun echo 'lets all love lain'}
  \item \textbf{\texttt{at}} -- утилита для работы с \texttt{atd} - службой,
    предназначенной для выполнения команд в заданное время.
    \begin{itemize}
      \item \textbf{\texttt{-l}} : показать список комманд, запланированных для
        исполнения
      \item \textbf{\texttt{-d}} : удалить команду из списка для исполнения
    \end{itemize}
    \texttt{echo 'lets all love lain' | at now + 1 minute}
  \item \textbf{\texttt{nice}} -- запустить программу с измененным приоритетом
    для планировщика задач. [-20, 20]
    \begin{itemize}
      \item \textbf{\texttt{-n}} : выставить niceness value
    \end{itemize}
    \texttt{nice -n -20 echo 'lets all love lain'}
  \item \textbf{\texttt{nohup}} -- запустить указанную команду с игнорированием
    сигналов о потере связи (\texttt{SIGHUP}), таким образом процесс продолжит
    жить даже после выхода пользователя из системы
    \texttt{nohup while true; do echo "lets all love lain"; sleep 1; done}
  \item \textbf{\texttt{kill}} -- послать сигнал указанным процессам
    \begin{itemize}
      \item \textbf{\texttt{-s}} : указать посылаемый сигнал (по имени или номеру)
      \item \textbf{\texttt{-l}} : вывести имена сигналов, поддерживаемых ОС
    \end{itemize}
    \texttt{kill -KILL 666}
  \item \textbf{\texttt{fg}} -- перевести задачу из фонового режима\\
    \texttt{fg 1}
  \item \textbf{\texttt{bg}} -- продолжить выполнение задачи в фоновом режиме\\
    \texttt{bg \%1}
  \item \textbf{\texttt{jobs}} -- отображает фоновые задачи текущей сессии.
    \begin{itemize}
      \item \textbf{\texttt{-l}} : отображать \texttt{PID} задач
      \item \textbf{\texttt{-n}} : отображать только задачи, изменившие свой
        статус с последнего отображения
      \item \textbf{\texttt{-p}} : отображать только \texttt{PID} задач
    \end{itemize}
    \texttt{jobs -l}
  \item \textbf{\texttt{priocntl}} -- отобразить или задать параметры планировщика
    для указанных процессов
    \begin{itemize}
      \item \textbf{\texttt{-l}} : отобразить классы, определенные в системе
      \item \textbf{\texttt{-d}} : отобразать текущие параметры планировщика,
        ассоциированные с процессами
      \item \textbf{\texttt{-s}} : задать пааметры планировщика для определенных
        процессов
      \item \textbf{\texttt{-e}} : выполнить указанную комманды с указанными
        классом и параметрами планировщика
    \end{itemize}
    \texttt{priocntl -l}
  \item \textbf{\texttt{\&}} -- запускает процесс в фоновом режиме
\end{itemize}

\section{Основные сигналы}
\begin{itemize}
  \item \textbf{\texttt{SIGTERM}} : сигнал завершения процесса, посылается командой
    kill по умолчанию
  \item \textbf{\texttt{SIGKILL}} : немедленное завершение, нельзя перехватить
    и обработать
  \item \textbf{\texttt{SIGTSTP}} : приостановка процесса, Ctrl-Z. Переводит
    процесс в режим остановки
  \item \textbf{\texttt{SIGSTOP}} : \texttt{SIGTSTP}, который не может быть
    проигнорирован или обработан
  \item \textbf{\texttt{SIGCONT}} : возобновление выполнения процесса
  \item \textbf{\texttt{SIGINT}} : сигнал прерывания, который посылается программе
    с помощью нажатия специальной комбинации клавиш для прерывания программы
    (обычно — Ctrl-C).
\end{itemize}

\end{document}
