\documentclass[12pt, a4paper]{article}
\usepackage[a4paper, includeheadfoot, mag=1000, left=2cm, right=1.5cm, top=1.5cm, bottom=1.5cm, headsep=0.8cm, footskip=0.8cm]{geometry}
% Fonts
\usepackage{fontspec, unicode-math}
\setmainfont[Ligatures=TeX]{CMU Serif}
\setmonofont{CMU Typewriter Text}
\usepackage[english, russian]{babel}
% Indent first paragraph
\usepackage{indentfirst}
\setlength{\parskip}{5pt}
% Diagrams
\usepackage{graphicx}
\usepackage{float}
% Source code
\usepackage{verbatim}
% Page headings
\usepackage{fancyhdr}
\pagestyle{fancy}
\renewcommand{\headrulewidth}{0pt}
\setlength{\headheight}{16pt}
%\newfontfamily\namefont[Scale=1.2]{Gloria Hallelujah}
\fancyhead{}
\begin{document}

% Title page
\begin{titlepage}
\begin{center}

\textsc{ФГАОУ ВО «Санкт-Петербургский национальный исследовательский университет информационных технологий, механики и оптики»\\[4mm]
Кафедра вычислительной техники}
\vfill
\textbf{Лабораторная работа №4\\[4mm]
Системное программное обеспечение\\[16mm]
}
\begin{flushright}
Саржевский Иван Анатольевич
\\[2mm]Группа P3202
\end{flushright}
\vfill
Санкт-Петербург\\[2mm]
2019 г.

\end{center}
\end{titlepage}

\section{Первое задание}
\begin{enumerate}
  \item \verb|grep "Sun" datebook|
  \item \verb|grep "^J" datebook|
  \item \verb|grep "700$" datebook|
  \item \verb|grep -v "834" datebook|
  \item \verb|grep ":12/" datebook|
  \item \verb|grep ":408-" datebook|
  \item \verb|grep -e "[A-Z][a-z][a-z][a-z][a-z], [A-z]" datebook|
  \item \verb|grep -e " [Kk][A-Za-z]*:" datebook|
  \item \verb|grep -e ":......$" datebook|
  \item \verb|grep "[Ll]incoln" datebook|
\end{enumerate}
\section{Второе задание}
\begin{enumerate}
  \item При выполнении команды \verb|grep '\<Tom\>' db| интерпретатор запускает
    утилиту grep передавая в качестве аргументов строки \verb|\<Tom\>| и db
    соответственно. Утилита grep распознает аргументы и ищет в файле db строки,
    содержащие слово Tom. Статус выполнения команды можно определить
    проанализировав код завершения утилиты grep, который сохраняется в
    переменной окружения \verb|$?| 
  \item При выполнении команды \verb|grep 'Tom Savage>' db| интерпретатор запускает
    утилиту grep передавая в качестве аргументов строки \verb|Tom Savage| и db
    соответственно. Утилита grep распознает аргументы и ищет в файле db строки,
    содержащие подстроку Tom Savage.
  \item При выполнении команды \verb|grep '^Tommy' db| интерпретатор запускает
    утилиту grep передавая в качестве аргументов строки \verb|^Tommy| и db
    соответственно. Утилита grep распознает аргументы и ищет в файле db строки,
    начинающиеся с \verb|Tommy|. 
  \item При выполнении команды \verb|grep '\.bak$' db| интерпретатор запускает
    утилиту grep передавая в качестве аргументов строки \verb|\.bak$| и db
    соответственно. Утилита grep распознает аргументы и ищет в файле db строки,
    оканчивающиеся на \verb|.bak|.
  \item При выполнении команды \verb|grep '[Pp]yramid' *| интерпретатор запускает
    утилиту grep передавая в качестве аргументов строки \verb|[Pp]yramid| и \verb|*|
    соответственно. Утилита grep распознает аргументы и ищет во всех файлах в директории строки,
    содержащие подстроку Pyramid или pyramid. 
  \item При выполнении команды \verb|grep '[A-Z]' db| интерпретатор запускает
    утилиту grep передавая в качестве аргументов строки \verb|[A-Z]| и db
    соответственно. Утилита grep распознает аргументы и ищет в файле db строки,
    содержащие заглавные буквы.
  \item При выполнении команды \verb|grep '[0-9]' db| интерпретатор запускает
    утилиту grep передавая в качестве аргументов строки \verb|[0-9]| и db
    соответственно. Утилита grep распознает аргументы и ищет в файле db строки,
    содержащие цифры.
  \item При выполнении команды \verb|grep '[A-Z]...[0-9]' db| интерпретатор запускает
    утилиту grep передавая в качестве аргументов строки \verb|[A-Z]...[0-9]| и db
    соответственно. Утилита grep распознает аргументы и ищет в файле db строки,
    содержащие подстроку, начинающуюся с заглавной буквы, затем три любых символа
    и цифра.
  \item При выполнении команды \verb|grep -w '[Tt]est' db| интерпретатор запускает
    утилиту grep передавая в качестве аргументов строки \verb|-w|, \verb|[A-Z]| и db
    соответственно. Утилита grep распознает аргументы и ищет в файле db строки,
    содержащие содержащие слова Test или test.
  \item При выполнении команды \verb|grep -s 'Mark Todd' db| интерпретатор запускает
    утилиту grep передавая в качестве аргументов строки \verb|-s|, \verb|Mark Todd| и db
    соответственно. Утилита grep распознает аргументы и ищет в файле db строки,
    содержащие подстроку \verb|Mark Todd| и подавляет сообщения о несуществующих
    или невалидных файлах.
  \item При выполнении команды \verb|grep -v 'Mary' db| интерпретатор запускает
    утилиту grep передавая в качестве аргументов строки \verb|-v|, \verb|Mary| и db
    соответственно. Утилита grep распознает аргументы и ищет в файле db строки,
    не содержащие подстроку \verb|Mary|.
  \item При выполнении команды \verb|grep -i 'sam' db| интерпретатор запускает
    утилиту grep передавая в качестве аргументов строки \verb|-i|, \verb|sam| и db
    соответственно. Утилита grep распознает аргументы и ищет в файле db строки,
    содержащие подстроку \verb|sam| без учета регистра (Sam, sAm, SaM etc).
  \item При выполнении команды \verb|grep -l 'Dear Boss' *| интерпретатор запускает
    утилиту grep передавая в качестве аргументов строки \verb|-l|, \verb|Dear Boss| и \verb|*|
    соответственно. Утилита grep распознает аргументы и выведет имена файлов,
    содержащих подстроку \verb|Dear Boss|.
  \item При выполнении команды \verb|grep -n 'Tom' db| интерпретатор запускает
    утилиту grep передавая в качестве аргументов строки \verb|-n|, \verb|Tom| и db
    соответственно. Утилита grep распознает аргументы и ищет в файле db строки,
    содержащие заглавные буквы, затем выведет их с указанием номера строки внутри файла.
  \item При выполнении команды \verb|grep "$name" db| интерпретатор запускает
    утилиту grep передавая в качестве аргументов строки \verb|$name| и db
    соответственно. Утилита grep распознает аргументы и ищет в файле db строки,
    содержащие подстроку из переменной окружения \verb|name|.
  \item При выполнении команды \verb|grep '$5' db| интерпретатор запускает
    утилиту grep передавая в качестве аргументов строки \verb|$5| и db
    соответственно. Утилита grep распознает аргументы и ищет в файле db строки,
    содержащие 5 после конца строки - ничего не найдет.
  \item При выполнении команды \verb|ps -ef  grep '^ *user1'| интерпретатор запускает
    утилиту grep передавая в качестве аргумента строку \verb|^ *user1|
    Утилита grep распознает аргументы и ищет в выводе команды \verb|ps -ef| строки,
    начинающиеся с произвольного количества пробелов, завершаемого подстрокой \verb|user1|.
    Так как команда \verb|ps -ef| вернет нам список процессов, данная команда
    выберет из них те, которые принадлежат пользователю \verb|user1|.
  \item При выполнении команды \verb|egrep '^ +' db| интерпретатор запускает
    утилиту egrep передавая в качестве аргументов строки \verb|^ +| и db
    соответственно. Утилита egrep распознает аргументы и ищет в файле db строки,
    начинающиеся с произвольного количества пробелов.
  \item При выполнении команды \verb|egrep '^ *' db| интерпретатор запускает
    утилиту egrep передавая в качестве аргументов строки \verb|^ *| и db
    соответственно. Утилита egrep распознает аргументы и ищет в файле db строки,
    начинающиеся с произвольного количества пробелов (включая 0).
  \item При выполнении команды \verb|egrep '(Tom Dan) Savage' db| интерпретатор запускает
    утилиту egrep передавая в качестве аргументов строки \verb|(Tom Dan) Savage| и db
    соответственно. Утилита egrep распознает аргументы и ищет в файле db строки,
    содержащие подстроки \verb|Tom Savage| или \verb|Dan Savage|.
  \item При выполнении команды \verb|egrep '(ab)+' db| интерпретатор запускает
    утилиту egrep передавая в качестве аргументов строки \verb|(ab)+| и db
    соответственно. Утилита egrep распознает аргументы и ищет в файле db строки,
    содержащие произвольное количество повторений подстроки \verb|ab| (ab, abab etc).
  \item При выполнении команды \verb|egrep '^X[0-9]?' db| интерпретатор запускает
    утилиту egrep передавая в качестве аргументов строки \verb|^X[0-9]?| и db
    соответственно. Утилита egrep распознает аргументы и ищет в файле db строки,
    начинающиеся с буквы \verb|X| и одного или двух чисел.
  \item При выполнении команды \verb|egrep 'fun\.$' *| интерпретатор запускает
    утилиту egrep передавая в качестве аргументов строки \verb|fun\.$| и \verb|*|
    соответственно. Утилита egrep распознает аргументы и ищет во всех файлах директории строки,
    оканчивающиеся на \verb|fun.|.
  \item При выполнении команды \verb|egrep '[A-Z]+' db| интерпретатор запускает
    утилиту egrep передавая в качестве аргументов строки \verb|[A-Z]+| и db
    соответственно. Утилита egrep распознает аргументы и ищет в файле db строки,
    содержажие произвольное количество подряд идущих заглавных букв.
  \item При выполнении команды \verb|egrep '[0-9]' db| интерпретатор запускает
    утилиту egrep передавая в качестве аргументов строки \verb|[0-9]| и db
    соответственно. Утилита egrep распознает аргументы и ищет в файле db строки,
    содержащие цифры.
  \item При выполнении команды \verb|egrep '[A-Z]...[0-9]' db| интерпретатор запускает
    утилиту egrep передавая в качестве аргументов строки \verb|[A-Z]...[0-9]| и db
    соответственно. Утилита egrep распознает аргументы и ищет в файле db строки,
    содержащие подстроку, начинающуюся с заглавной буквы, затем три произвольных
    символа и цифра.
  \item При выполнении команды \verb|egrep '[tT]est' db| интерпретатор запускает
    утилиту egrep передавая в качестве аргументов строки \verb|[tT]est| и db
    соответственно. Утилита egrep распознает аргументы и ищет в файле db строки,
    содержащие подстроки \verb|Test| или \verb|test|.
  \item При выполнении команды \verb|egrep '(Suesan Jean) Doe' db| интерпретатор запускает
    утилиту egrep передавая в качестве аргументов строки \verb|(Suesan Jean) Doe| и db
    соответственно. Утилита egrep распознает аргументы и ищет в файле db строки,
    содержащие подстроки \verb|Suesan Doe| или \verb|Jean Doe|.
  \item \verb|egrep -v 'Mary' db| - аналогично пункту 11.
  \item \verb|egrep -i 'sam' db| - аналогично пункту 12.
  \item \verb|egrep -l 'Dear Boss' db| - аналогично пункту 13.
  \item \verb|egrep -n 'Tom' db| - аналогично пункту 14.
  \item \verb|egrep -s "$name" db| - аналогично пункту 15, с подавлением сообщений
    об ошибках.
\end{enumerate}
\end{document}
