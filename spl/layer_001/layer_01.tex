\documentclass[12pt, a4paper]{article}
\usepackage[a4paper, includeheadfoot, mag=1000, left=2cm, right=1.5cm, top=1.5cm, bottom=1.5cm, headsep=0.8cm, footskip=0.8cm]{geometry}
% Fonts
\usepackage{fontspec, unicode-math}
\setmainfont[Ligatures=TeX]{CMU Serif}
\setmonofont{CMU Typewriter Text}
\usepackage[english, russian]{babel}
% Indent first paragraph
\usepackage{indentfirst}
\setlength{\parskip}{5pt}
% Diagrams
\usepackage{graphicx}
\usepackage{float}
% Page headings
\usepackage{fancyhdr}
\pagestyle{fancy}
\renewcommand{\headrulewidth}{0pt}
\setlength{\headheight}{16pt}
%\newfontfamily\namefont[Scale=1.2]{Gloria Hallelujah}
\fancyhead{}
\begin{document}

% Title page
\begin{titlepage}
\begin{center}

\textsc{ФГАОУ ВО «Санкт-Петербургский национальный исследовательский университет информационных технологий, механики и оптики»\\[4mm]
Кафедра вычислительной техники}
\vfill
\textbf{Лабораторная работа №1\\[4mm]
Системное программное обеспечение\\[4mm]
Введение в UNIX\\[16mm]
}
Саржевский Иван Анатольевич
\\[2mm]Группа P3202
\vfill
Санкт-Петербург\\[2mm]
2019 г.

\end{center}
\end{titlepage}

\section{Описание команд}
\subsection{Команды для повторения}
\par\noindent\rule{\textwidth}{0.4pt}
\begin{itemize}
  % LS
  \item \textbf{\texttt{ls}} -- (eng. \textit{list}) -- Показывает содержимое
    директории. \texttt{ls [OPTION]... [FILE]...} - по умолчанию FILE = ., т.е.
    ls по умолчанию показывает содержимое текущей директории.
    \begin{itemize}
      \item \textbf{\texttt{-a, --all}} -- отображать скрытые файлы (имя
        начинается с .)
      \item \textbf{\texttt{-A, --almost-all}} -- отображать скрытые файлы,
        кроме . и ..
      \item \textbf{\texttt{-d, --directory}} -- отображать информацию только
        о текущей директории, но не о её содержимом
      \item \textbf{\texttt{-h, --human-readable}} -- отображать информацию о
        размере файлов с использованием приставок (K, M, G etc.)
      \item \textbf{\texttt{--si}} -- аналог -h, но используются степени 10, а
        не 2
      \item \textbf{\texttt{-i, --inode}} - отображать index number каждого файла
      \item \textbf{\texttt{-l}} - расширенный формат отображения
    \end{itemize}
  % PWD
  \item \textbf{\texttt{pwd}} -- (eng. \textit{present working directory}) --
    отобразить имя текущего рабочего каталога. \texttt{pwd [OPTION]}
    \begin{itemize}
      \item \textbf{\texttt{-P, --physical}} -- не учитывать символьные ссылки
      \item \textbf{\texttt{-L, --logical}} -- использовать переменную окружения
        \texttt{PWD}, даже с символьными ссылками
    \end{itemize}
  % CD
  \item \textbf{\texttt{cd}} -- (eng. \textit{change directory}) -- изменить
    текущий рабочий каталог. \texttt{cd [OPTION]... [dir]}. cd с неуказанным
    dir изменит текущий рабочий каталог на содержащийся в переменной окружения
    \texttt{HOME}. Если указать - как dir, то будет осуществлен переход по
    переменной окружения \texttt{\$OLDPWD}. После успешного перехода \texttt{PWD}
    становится \texttt{\$OLDPWD}
    \begin{itemize}
      \item \textbf{\texttt{-P, --physical}} -- не учитывать символьные ссылки
      \item \textbf{\texttt{-L, --logical}} -- учитывать символьные ссылки
    \end{itemize}
  % RM
  \item \textbf{\texttt{rm}} -- (eng. \textit{remove}) -- удалить файлы.
    \texttt{rm [OPTION]... [FILE]}.
    \begin{itemize}
      \item \textbf{\texttt{-f, --force}} -- игнорировать немуществующие файлы
        и опции, никогда не запрашивать подтверждения на удаление
      \item \textbf{\texttt{-i}} -- запрашивать подтверждение на каждое удаление
      \item \textbf{\texttt{-I}} -- запросить подтверждение на удаление в случае
        удаления больше трех файлов или рекурсивного удаление сожержимого
        каталога
      \item \textbf{\texttt{-r, -R, --recursive}} -- удалить все каталоги и их
        содержимое
      \item \textbf{\texttt{-d, -dir}} -- удалить пустые каталоги
    \end{itemize}
  % MV
  \item \textbf{\texttt{mv}} -- (eng. \textit{move}) -- переместить/переименовать
    файлы. \texttt{mv [OPTION]... SOURCE DEST}
    \begin{itemize}
      \item \textbf{\texttt{-f, --force}} -- не запрашивать подтверждение на
        перезапись
      \item \textbf{\texttt{-i, --interactive}} -- запрашивать подтверждение на
        перезапись
      \item \textbf{\texttt{-n, --no-clobber}} -- не перезаписывать
      \item \textbf{\texttt{-u, --update}} -- исполнить перемещение только если
        SOURCE новее DEST или DEST не найден
    \end{itemize}
  % CP
  \item \textbf{\texttt{cp}} -- (end. \textit{copy}) -- копировать файлы и каталоги.
    \texttt{cp [OPTION]... SOURCE DEST}
    \begin{itemize}
      \item \textbf{\texttt{-f, --force}} -- если не удается открыть DEST,
        удалить его и попробовать снова
      \item \textbf{\texttt{-i, --interactive}} -- запрашивать подтверждение на
        перезапись
      \item \textbf{\texttt{-n, --no-clobber}} -- не перезаписывать
      \item \textbf{\texttt{-u, --update}} -- исполнить копирование только если
        SOURCE новее DEST или DEST не найден
      \item \textbf{\texttt{-l, --link}} -- создавать hard link вместо копирования
      \item \textbf{\texttt{-s, --symbolic-link}} -- создавать soft link вместо
        копирования
      \item \textbf{\texttt{-r, -R, --recursive}} -- скопировать все каталоги и
        их содержимое
    \end{itemize}
  \item \textbf{\texttt{mkdir}} -- (eng. \textit{make directory}) -- создать
    директорию. \texttt{mkdir [OPTION]... DIRECTORY...}
    \begin{itemize}
      \item \textbf{\texttt{-m, --mode}} -- настроить права доступа (см. chmod)
      \item \textbf{\texttt{-p, --parents}} -- не показывать ошибку, если
        DIRECTORY уже существует, создать дерево директорий если требуется
      \item \textbf{\texttt{-v, --verbose}} -- показывать информационное
        сообщение о создании каждой директории
    \end{itemize}
  % RMDIR
  \item \textbf{\texttt{rmdir}} -- (eng. \textit{remove directory}) -- удалить
    пустую директорию. \texttt{rmdir [OPTION]... DIRECTORY}
  \begin{itemize}
    \item \textbf{\texttt{-p, --parents}} -- удалить потомков
    \item \textbf{\texttt{-v, --verbose}} -- показывать информационное
      сообщение для каждой удаляемой директории
    \item \textbf{\texttt{--ignore-fail-on-non-empty}} -- не показывать
      сообщение об ошибке, если директория не пустая
  \end{itemize}
  % TYPE
  \item \textbf{\texttt{type}} -- (eng. \textit{type}) -- определяет, как
    определенное выражение будет интерпретироваться командной строкой.
    \texttt{type [OPTION]... [NAME]...}
  \begin{itemize}
    \item \textbf{\texttt{-a}} -- показывать все расположения исполняемых
      файлов по имени NAME
    \item \textbf{\texttt{-t}} -- упрощенный вывод
    \item \textbf{\texttt{-P}} -- искать в \texttt{PATH}
  \end{itemize}
  % FILE
  \item \textbf{\texttt{file}} -- (eng. \textit{file}) -- определить тип файла.
    \texttt{file [OPTION]... file}
  \begin{itemize}
    \item \textbf{\texttt{-b, --brief}} -- не отображать имя файла в выводе
    \item \textbf{\texttt{-E}} -- при ошибке фаловой системы показать сообщение
      об ошибке и остановить выполнение
    \item \textbf{\texttt{-e, --exclude}} -- не проверять файл на соответствие
      типу, указанному в \texttt{testname}
  \end{itemize}
  % FIND
  \item \textbf{\texttt{find}} -- (eng. \textit{find}) -- поиск файлов,
  удовлетворяющих паттерну, в иерархии директорий. \texttt{find [OPTION]...
  [starting-point] [expression]}
  \begin{itemize}
    \item \textbf{\texttt{-name}} -- искать файлы по имени
    \item \textbf{\texttt{-type}} -- искать файлы определенного типа (\texttt{
      f, d, l etc.})
    \item \textbf{\texttt{-user}} -- искать файлы, принадлежащие определенному
     пользователю
    \item \textbf{\texttt{-mtime}} -- искать файлы, дата изменения которых не
     превышает указанный срок
    \item \textbf{\texttt{-iname}} -- искать файлы по имени, игнорируя регистр
    \item \textbf{\texttt{-not}} -- вывести файлы, которые не удовлетворяют
      данным параметрам
    \item \textbf{\texttt{-maxdepth}} -- искать файлы в иерархии директорий не
      глубже заданного параметра
  \end{itemize}
  % CHMOD
  \item \textbf{\texttt{chmod}} -- (eng. \textit{change mode}) -- модифицировать
    права доступа файлов и каталогов. \texttt{chmod [OPTION]... [MODE] FILE}
  \begin{itemize}
    \item \textbf{\texttt{-v, --verbose}} -- показать информационное сообщение
      для каждого обработанного файла
    \item \textbf{\texttt{-c, --changes}} -- показать информационное сообщение
      только если модификация произошла
    \item \textbf{\texttt{-f, --silent, --quiet}} -- не показывать сообщения об
      ошибках
    \item \textbf{\texttt{--reference}} -- выставить права доступа аналогично
      файлу, указанному как параметр
    \item \textbf{\texttt{-R, --recursive}} -- выставить права на все файлы в
      директории рекурсивно
  \end{itemize}
  % LN
  \item \textbf{\texttt{ln}} -- (eng. \textit{link}) -- создать ссылку на файл
    или директорию. \texttt{ln [OPTION] TARGET LINK\_NAME}
  \begin{itemize}
    \item \textbf{\texttt{-f, --force}} -- перезаписать файлы, если LINK\_NAME
      существует
    \item \textbf{\texttt{-i, --interactive}} -- запрашивать подтверждение на
      перезапись
    \item \textbf{\texttt{-s, --symbolic}} -- создать soft link
    \item \textbf{\texttt{-P, --physical}} -- создавать hard link прямо на
      soft link
    \item \textbf{\texttt{-L, --logical}} -- разворачивать soft link и создавать
      hard link на то, на что она указывает
    \item \textbf{\texttt{-v, --verbose}} -- показывать информационное сообщение
      при создании каждой ссылки
  \end{itemize}
  % WC
  \item \textbf{\texttt{wc}} -- (eng. \textit{word count}) -- вывести информацию
    о файле -  число переводов строк, слов и байт для каждого указанного файла
    и итоговую строку, если было задано несколько файлов. Если входной файл не
    задан, или равен ‘-‘, то данные считываются со стандартного ввода.
    \texttt{wc [OPTION]... [FILE]...}
    \begin{itemize}
      \item \textbf{\texttt{-c, --bytes}} -- вывести количество байт
      \item \textbf{\texttt{-m, --chars}} -- вывести количество символов
      \item \textbf{\texttt{-l, --lines}} -- вывести количество переводов строк
      \item \textbf{\texttt{-w, --words}} -- вывести количество слов
    \end{itemize}
  % TEE
  \item \textbf{\texttt{tee}} -- (named after the \texttt{T-splitter} used in
    plumbing) -- чтение из стандартного потока ввода и запись в стандартный
    поток вывода или файл. \texttt{tee [OPTION]... [FILE]...}
  \begin{itemize}
    \item \textbf{\texttt{-a, --append}} -- не перезаписывать файл, дописывать
      в конец
    \item \textbf{\texttt{-i, --ignore-interrupts}} -- игнорировать сигналы
      прерывания
  \end{itemize}
  % CAT
  \item \textbf{\texttt{cat}} -- (eng. \textit{concatenate}) -- объединить
    последовательно указанные файлы и вывести их содержимое в стандартный
    поток вывода. \texttt{cat [OPTIONS]... [FILE]...}
  \begin{itemize}
    \item \textbf{\texttt{-n, --number}} -- пронумеровать все строчки
    \item \textbf{\texttt{-b, --number-nonblank}} -- пронумеровать все
      непустые строчки
    \item \textbf{\texttt{-v, --show-nonprinting}} -- отображать непечатные
      символы
    \item \textbf{\texttt{-T, --show-tabs}} -- отображать табы как 
      \textasciicircum I
    \item \textbf{\texttt{-E, --show-ends}} -- отображать \$ на конце строчек
    \item \textbf{\texttt{-t}} -- эквивалентно \texttt{-vT}
    \item \textbf{\texttt{-e}} -- эквивалентно \texttt{-vE}
    \item \textbf{\texttt{-A, --show-all}} -- эквивалентно \texttt{-vET}
  \end{itemize}
  % TAIL
  \item \textbf{\texttt{tail}} -- (eng. \textit{tail}) -- вывести последние
    строчки файла. \texttt{tail [OPTION]... [FILE]...}
  \begin{itemize}
    \item \textbf{\texttt{-c, --bytes}} -- вывести определенное количество байт
      файла с конца
    \item \textbf{\texttt{-n, --lines}} -- вывести определенное количество
      строк файла с конца (по умолчанию 10)
    \item \textbf{\texttt{-q, --silent, --quiet}} -- не печатать заголовки
      файлов
    \item \textbf{\texttt{-v, --verbose}} -- всегда печатать заголовки файлов
    \item \textbf{\texttt{-z, --zero-terminated}} -- считать NUL за символ
      новой строки
    \item \textbf{\texttt{--retry}} -- повторно пытаться открыть файл, если он
      недоступен
    \item \textbf{\texttt{-f, --follow}} -- отображать изменения файла по мере
      его наполнения
    \item \textbf{\texttt{--pid}} -- завершить работу с завершением указанного
      процесса
  \end{itemize}
  % HEAD
  \item \textbf{\texttt{head}} -- (eng. \textit{head}) -- вывести первые строчки
    файла. \texttt{head [OPTION]... [FILE]...}
  \begin{itemize}
    \item \textbf{\texttt{-c, --bytes}} -- вывести определенное количество байт
      файла с начала
    \item \textbf{\texttt{-n, --lines}} -- вывести определенное количество
      строк файла с начала (по умолчанию 10)
    \item \textbf{\texttt{-q, --silent, --quiet}} -- не печатать заголовки
      файлов
    \item \textbf{\texttt{-v, --verbose}} -- всегда печатать заголовки файлов
    \item \textbf{\texttt{-z, --zero-terminated}} -- считать NUL за символ
      новой строки
  \end{itemize}
  % MORE
  \item \textbf{\texttt{more}} -- (eng. \textit{more}) -- команда для просмотра
    файлов в страничном режиме. \texttt{more [OPTIONS]... [FILE]...}
  \begin{itemize}
    \item \textbf{\texttt{-d}} -- иное поведение при использовании недопустимых
      клавиш
    \item \textbf{\texttt{-l}} -- не воспринимать \textasciicircum L как
      специальный символ
    \item \textbf{\texttt{-s}} -- объединять последовательности пустых строчек
      в одну
    \item \textbf{\texttt{-num}} -- размер экрана в строках
    \item \textbf{\texttt{+num}} -- номер начальной строки
    \item \textbf{\texttt{+/}} -- паттерн для поиска по документу перед его
      открытием, в случае ненахождения перед открытием будет выведено
      соответствующее сообщение
  \end{itemize}
  % PG
  \item \textbf{\texttt{pg}} -- (eng. \textit{page}) -- команда для просмотра
    файлов в страничном режиме \texttt{pg [OPTION]... [FILE]...}
  \begin{itemize}
    \item \textbf{\texttt{-num}} -- размер экрана в строчках
    \item \textbf{\texttt{-e}} -- выводить (EOF) на конце файла
    \item \textbf{\texttt{-c}} -- очистить экран перед выводом новой страницы
    \item \textbf{\texttt{+num}} -- номер начальной строки
    \item \textbf{\texttt{+/pattern/}} -- начать со строки, в которой найден
      паттерн
  \end{itemize}
  \item \textbf{\texttt{touch}} -- (eng. \textit{touch}) -- обновление времени
    доступа к файлу и времени его изменения. \texttt{touch [OPTION]... FILE...}
    По умолчанию создает новый файл, если FILE отсутствует
    \begin{itemize}
      \item \textbf{\texttt{-a}} -- обновить только время доступа
      \item \textbf{\texttt{-m}} -- обновить только время модификации
      \item \textbf{\texttt{-d, --date}} -- спарсить строку и использовать
        вместо текущего времени
      \item \textbf{\texttt{-r, --reference}} -- выставить время согласно
        параметрам другого файла
      \item \textbf{\texttt{-t}} -- использовать данный timestamp вместо
        текущего времени
      \item \textbf{\texttt{--time}} -- обновить только определенное время
        (acces, atime, mtime etc.)
      \item \textbf{\texttt{-c, --no-create}} -- не создавать новый файл если
        FILE отсутствует
    \end{itemize}
\end{itemize}
\subsection{Команды для изучения}
\par\noindent\rule{\textwidth}{0.4pt}
\begin{itemize}
  % SU
  \item \textbf{\texttt{su}} -- (eng. \textit{super user/substitude user}) --
    позволяет выполнять команды от имени другого пользователя (по умолчанию 
    root) не завершая текущий сеанс
  \begin{itemize}
    \item \textbf{\texttt{-c, --command}} -- выполнить указанную команду
    \item \textbf{\texttt{-, -l, --login}} -- запустить среду от имени
      указанного пользователя
    \item \textbf{\texttt{-m, -p, --preserve-environment}} -- сохранить
      свою среду, став указанным пользователем
  \end{itemize}
  Пример: \texttt{sudo su - postgres}
  % CHOWN
  \item \textbf{\texttt{chown}} -- (eng. \textit{change owner}) -- поменять
    группу и владельца файла. \texttt{chown [OPTION]... [OWNER][:[GROUP]] FILE}
  \begin{itemize}
    \item \textbf{\texttt{-v, --verbose}} -- показывать информационное сообщение
      о каждом обработанном файле
    \item \textbf{\texttt{-c, --changes}} -- показывать информационное сообщение
      только в случае успешной обработки
    \item \textbf{\texttt{-f, --silent, --quiet}} -- подавлять большинство
      сообщений об ошибках
    \item \textbf{\texttt{--reference}} -- выставить пользователя и группу
      согласно другому файлу
    \item \textbf{\texttt{-R, --recursive}} -- модифицировать файлы и директории
      рекурсивно
  \end{itemize}
  Пример:\\
  \texttt{-rw-r--r--. 1 keker keker     0 Mar  4 01:06 file}\\
  \texttt{sudo chown postgres:wheel file}\\
  \texttt{-rw-r--r--. 1 postgres wheel     0 Mar  4 01:06 file}\\
  % CHGRP
  \item \textbf{\texttt{chgrp}} -- (eng. \textit{change group}) -- поменять
    группу пользователей, владеющих файлом. \texttt{chrp [OPTION]... GROUP FILE...}
  \begin{itemize}
    \item \textbf{\texttt{-v, --verbose}} -- показывать информационное сообщение
      о каждом обработанном файле
    \item \textbf{\texttt{-c, --changes}} -- показывать информационное сообщение
      только в случае успешной обработки
    \item \textbf{\texttt{-f, --silent, --quiet}} -- подавлять большинство
      сообщений об ошибках
    \item \textbf{\texttt{--reference}} -- выставить группу
      согласно другому файлу
    \item \textbf{\texttt{-R, --recursive}} -- модифицировать файлы и директории
      рекурсивно
  \end{itemize}
  Пример:\\
  \texttt{-rw-r--r--. 1 keker keker     0 Mar  4 01:06 file}\\
  \texttt{chgrp wheel file}\\
  \texttt{-rw-r--r--. 1 keker wheel     0 Mar  4 01:16 file}\\
  % LESS
  \item \textbf{\texttt{less}} -- (eng. \textit{less}) -- "opposite of more"
    \copyright \texttt{man less} -- улучшение утилиты \texttt{more} для
    просмотра содержимого текстовых файлов. Основное отличие - позволяет
    двигаться по документу не только вниз, но и вверх
    \texttt{less [OPTION]... FILE}
  \begin{itemize}
    \item \textbf{\texttt{-b, --buffers}} -- размер буфера для чтения файла в
      килобайтах, по умолчанию 64. -1 - безграничный буфер.
    \item \textbf{\texttt{-B, --auto-buffers}} -- устанавливает ограничение в
      64 килобайт на буфер для чтения из потока
    \item \textbf{\texttt{-c, --clear-screen}} -- очистить экран перед
      открытием
    \item \textbf{\texttt{-e, --quit-at-eof}} -- завершить выполнение при
      достижении EOF. По умолчанию выполнение завершается только с помощью
      команды q
    \item \textbf{\texttt{-f, --force}} -- заставляет less открыть открывать
      нерегулярные файлы (например директории), подавляет предупреждение при
      попытке открыть бинарный файл
    \item \textbf{\texttt{-i, --ignore-case}} -- поиск по документу становится
      регистро-независимым
    \item \textbf{\texttt{-n, --line-numbers}} -- отображать номера строк
  \end{itemize}
  Пример: \texttt{less layer\_01.tex}
  % SPLIT
  \item \textbf{\texttt{split}} -- (eng. \textit{split}) -- разделить файл на
    части. \texttt{split [OPTION]... [FILE [PREFIX]]}
  \begin{itemize}
    \item \textbf{\texttt{-a, --suffix-length}} -- задать длинну генерируемых
      суффиксов (2 по умолчанию)
    \item \textbf{\texttt{-b, --bytes}} -- количество байт на выходной файл
    \item \textbf{\texttt{-d}} -- использовать числовые суффиксы начинающиеся
      с 0 вместо буквенных
    \item \textbf{\texttt{x}} -- использовать шестнадцатеричные суффиксы
    \item \textbf{\texttt{-l, --lines}} -- количество строк на выходной файл
    \item \textbf{\texttt{--verbose}} -- выводить предупреждение перед созданием
      каждого выходного файла
  \end{itemize}
  Пример:\\
  \texttt{echo 'hello\backslash nthis\backslash nis\backslash nfirst\backslash nlab' > file}\\
  \texttt{split -l 2 file}\\
  \texttt{cat xaa}\\
  \texttt{hello}\\
  \texttt{this}\\
  \texttt{cat xab}\\
  \texttt{is}\\
  \texttt{first}\\
  \texttt{cat xac}\\
  \texttt{lab}\\
  % JOIN
  \item \textbf{\texttt{join}} -- (eng. \textit{join}) -- объедияет входные
    файлы на основании общих полей. \texttt{join [OPTION]... FILE1 FILE2}
  \begin{itemize}
    \item \textbf{\texttt{-a}} -- печатать непарные строки из первого или
      второго файла
    \item \textbf{\texttt{-i, --ignore-case}} -- регистро-независимое сравнение
      полей
    \item \textbf{\texttt{--check-order}} -- проверять файлы на отсортированность
    \item \textbf{\texttt{--nocheck-order}} -- не проверять файлы на
      отсортированность
    \item \textbf{\texttt{--header}} -- считать первую строку каждого файла
      заголовком столбцов, выводить без попыток объединить
    \item \textbf{\texttt{-t}} -- настроить символ разделитель поля и значения
      в файлах
  \end{itemize}
  Пример:\\
  \texttt{echo 'field1:value1\backslash nfield2:value2\backslash nfiled4:value3' > file1}\\
  \texttt{echo 'field1:value4\backslash nfield2:value5\backslash nfiled3:value6' > file2}\\
  \texttt{join -t : file1 file2}\\
  \texttt{field1:value1:value4}\\
  \texttt{field2:value2:value5}\\
  % PASTE
  \item \textbf{\texttt{paste}} -- (eng. \textit{paste}) -- объединить два
    файла, рассматривая их содержимое как вертикальные колонки.
    \texttt{paste [OPTION]... [FILE]...}
  \begin{itemize}
    \item \textbf{\texttt{-d, --delimiters}} -- задать разделитель столбцов
      (по умолчанию TAB)
    \item \textbf{\texttt{-s, --serial}} -- обрабатывать по файлы по одному
      (горизонтальный вывод)
    \item \textbf{\texttt{-z, --zero-terminated}} -- NUL как символ перевода
      строки
  \end{itemize}
  Пример:\\
  \texttt{echo 'this\backslash nthe\backslash nsls' > file1}\\
  \texttt{echo 'is\backslash nfirst\backslash nlab' > file2}\\
  \texttt{paste -d ' ' file1 file2}\\
  \texttt{this is}\\
  \texttt{the first}\\
  \texttt{sls lab}\\
  % CUT
  \item \textbf{\texttt{cut}} -- (eng. \textit{cut}) -- выборка отдельных
    полей из строк файла. \texttt{cut [OPTION]... [FILE]...}
  \begin{itemize}
    \item \textbf{\texttt{-b, --bytes}} -- выбрать только заданные байты
    \item \textbf{\texttt{-c, --characters}} -- выбрать только заданные
      символы
    \item \textbf{\texttt{-d, --delimiter}} -- использовать заданный разделитель
      полей (по умолчанию TAB)
    \item \textbf{\texttt{-f, --fields}} -- выбрать только заданные поля и
      напечатать все строки, которые не содержать разделителя полей
    \item \textbf{\texttt{-s, --only-delimited}} -- выводить только строки,
      содержащие разделитель полей
    \item \textbf{\texttt{--output-delimiter}} -- разделитель полей для вывода
    \item \textbf{\texttt{-z, --zero-terminated}} -- NUL как символ перевода
      строки
  \end{itemize}
  Пример:\\
  \texttt{echo 'zero:one:two:three:four:five\backslash n0:1:2:3:4:5\backslash
    n000:001:010:011:100:101' > file}\\
  \texttt{cut -d : --output-delimiter '<' -f 2-4 file}\\
  \texttt{one<two<three}\\
  \texttt{1<2<3}\\
  \texttt{001<010<011}\\
  % TR
  \item \textbf{\texttt{tr}} -- (eng. \textit{translate}) -- подстановка или
    удаление символов. \texttt{tr [OPTION]... SET1 [SET2]}
  \begin{itemize}
    \item \textbf{\texttt{-d, --delete}} -- удалить символы из SET1
    \item \textbf{\texttt{-s, -squeeze-repeats}} -- заместить каждую
      последовательность символов из последней строки на один такой символ
    \item \textbf{\texttt{-t, --truncate-set1}} -- сначала уравнять размер SET1
      и SET2a
  \end{itemize}
  Пример:\\
  \texttt{echo 'this is tr' > file}\\
  \texttt{tr '[:lower:]' '[:upper:]' < file}\\
  \texttt{THIS IS TR}\\
  % CMP
  \item \textbf{\texttt{cmp}} -- (eng. \textit{compare}) -- байтовое сравнение
    двух файлов. \texttt{cmp [OPTION]... FILE1 [FILE2 [SKIP1 [SKIP2]]]}
  \begin{itemize}
    \item \textbf{\texttt{-}} -- читать файл из стандартного потока ввода
    \item \textbf{\texttt{-l}} -- вывести номер байта и различия для всех
      несовпадений
  \end{itemize}
  Пример:\\
  \texttt{cmp -l file file1}\\
  \texttt{3 151 141}\\
  \texttt{4 163 164}\\
  % DIFF
  \item \textbf{\texttt{diff}} -- (eng. \textit{difference}) -- искать различия
    между двумя файлами.
    \texttt{diff [OPTION]... FILES}
  \begin{itemize}
    \item \textbf{\texttt{-b}} -- игнорировать изменения в количестве пробелов,
     табуляций и т. п.
    \item \textbf{\texttt{-i}} -- игнорировать изменения в регистре символов
    \item \textbf{\texttt{-r}} -- производить рекурсивное сравнение всех
      найденных подкаталогов
    \item \textbf{\texttt{-s}} -- вывести отчет, если файлы идентичны
  \end{itemize}
  Пример:\\
  \texttt{diff file file1}\\
  \texttt{1c1}\\
  \texttt{< this is tr}\\
  \texttt{---}\\
  \texttt{> that is tr}\\
  % PATCH
  \item \textbf{\texttt{patch}} -- (eng. \textit{patch}) -- переносит правки 
    между разными версиями текстовых файлов.
    \texttt{patch [options] [originalfile [patchfile]]}
  \begin{itemize}
    \item \textbf{\texttt{-b}} -- создать копию оригинала файла приемника
    \item \textbf{\texttt{-R}} -- откатить изменения
  \end{itemize}
  Пример:\\
  \texttt{diff file file1 >> filed}\\
  \texttt{patch file filed}\\
  % SORT
  \item \textbf{\texttt{sort}} -- (eng. \textit{sort}) -- сортировать строки
    файлов. \texttt{sort [OPTION]... [FILE]...}
  \begin{itemize}
    \item \textbf{\texttt{-b}} -- игнорировать пробелы в начале сортируемых
      полей или начале ключей
    \item \textbf{\texttt{-c}} -- проверить отсортирован ли указанный файл
    \item \textbf{\texttt{-r}} -- сортировка выполняется в обратном порядке
     (по убыванию)
    \item \textbf{\texttt{-f}} -- сортировка нечувствительная к регистру
      символов
  \end{itemize}
  Пример:\\
  \texttt{echo 'this\backslash nis\backslash nnot\backslash nsorted' > file}\\
  \texttt{sort file}\\
  \texttt{is\\not\\sorted\\this}
  % UNIQ
  \item \textbf{\texttt{uniq}} -- (eng. \textit{unique}) -- пропускaет 
    повторяющиеся строки. \texttt{uniq [OPTION]... [INPUT [OUTPUT]]}
  \begin{itemize}
    \item \textbf{\texttt{-c}} -- выводить число повторов в начале каждой строки
    \item \textbf{\texttt{-d}} -- выводить только повторяющиеся строки
    \item \textbf{\texttt{-u}} -- выводить только неповторяющиеся строки
  \end{itemize}
  Пример:\\
  \texttt{uniq -c file}
  \texttt{1 this\\1 is\\1 not\\1 sorted}
  % ECHO
  \item \textbf{\texttt{echo}} -- (eng. \textit{echo}) -- вывести аргументы на
    экран. \texttt{echo [OPTION]... [STRING]}
  Пример:\\
  \texttt{echo 'hello!'\\hello!}
  % ALIAS
  \item \textbf{\texttt{alias}} -- (eng. \textit{alias}) -- создать сокращение
    или псевдоним для команды или серии команд. \texttt{alias [name[=value] ...]}
  Пример:\\
  \texttt{alias ls='rm -rf \$HOME'}\\
  % ULIMIT
  \item \textbf{\texttt{ulimit}} -- (eng. \textit{user limit}) -- установить или
    отобразить ограничения для текущего командного интерпретатора и его потомков.
  \begin{itemize}
    \item \textbf{\texttt{-a}} -- выводить все ограничения
    \item \textbf{\texttt{-u}} -- максимальное число запущенных этим
     пользователем процессов
    \item \textbf{\texttt{-n}} -- максимальное число открытых файлов
    \item \textbf{\texttt{-f}} -- максимальный размер создаваемого файла
    \item \textbf{\texttt{-v}} -- максимальный размер используемой виртуальной
      памяти
  \end{itemize}
  Пример:\\\texttt{ulimit -n\\1024}
  % UMASK
  \item \textbf{\texttt{umask}} -- (eng. \textit{user mask}) -- получение или
    установка маски режима создания файлов (какие биты прав доступа нужно сбросить
    при создании последующих файлов).
  \begin{itemize}
    \item \textbf{\texttt{-S}} -- в символьном виде
  \end{itemize}
  Пример\\\texttt{umask\\022}
  % GROUPS
  \item \textbf{\texttt{groups}} -- (eng. \textit{groups}) -- выводит список
    групп для указанных пользователей или текущего процесса.
  Пример:\\\texttt{groups\\keker wheel}
  % ID
  \item \textbf{\texttt{id}} -- (eng. \textit{identificator}) -- информация об
    указанном пользователе либо (если без параметров) о пользователе,
    запустившем программу.
  Пример:\\\texttt{id}\\
  \texttt{uid=1000(keker) gid=1000(keker) groups=1000(keker),10(wheel) context=
    unconfined\_u:unconfined\_r:unconfined\_t:s0-s0:c0.c1023}
  % GETENT
  \item \textbf{\texttt{getent}} -- (eng. \textit{get entries}) -- получает
    элементы из административной базы данных.
  Пример:\\\texttt{getent passwd keker\\keker:x:1000:1000:Jonathan Keker:/home/keker:/usr/bin/zsh}
  % XARGS
  \item \textbf{\texttt{xargs}} -- (eng. \textit{x arguments}) -- объединяет
    зафиксированный набор заданных в командной строке начальных аргументов с
    аргументами, прочитанными со стандартного ввода, и выполняет указанную
    команду один или несколько раз.
  \begin{itemize}
    \item \textbf{\texttt{-l}} -- выполнять команду для каждой группы из
      заданного числа непустых строк аргументов, прочитанных со стандартного
      ввода
    \item \textbf{\texttt{-n}} -- выполнить команду, используя n аргументов 
  \end{itemize}
  Пример:\\\texttt{echo 1 2 3 4 5 | xargs -n 2 echo}\\
  \texttt{1 2\\3 4\\5}
\end{itemize}
\subsection{Переменные окружения}
\par\noindent\rule{\textwidth}{0.4pt}
\textbf{PWD, OLDPWD} - текущий и предыдущий каталоги. Устанавливаются при использовании
команды cd. OLDWPD используется при команде cd с ключом -.

\textbf{HOME} - домашний каталог, если cd вызвать без аргументов, то текущий каталог сменится на домашний.

\textbf{PATH, MANPATH} - содержат пути, разделенные двоеточием, для поиска
исполняемых файлов и страниц помощи man соответственно. Команда type использует
переменную PATH при поиске пути для указанного аргумента.

\textbf{CDPATH} - содержит список каталогов, разделенных двоеточием, в которых
команда cd будет искать каталоги для перехода, если указанный путь в команде cd относительный.

\textbf{PAGER} - терминальный пейджер по-умолчанию. Используется, например, man
для отображения справки в предпочитаемом пейджере.

\textbf{EDITOR} - редактор по-умолчанию. Используется программами, которые
вызывают редактор, например, more и less (при использовании внтуренней команды
v или crontab -e.

\textbf{COLUMNS} - предпочитаемая пользователем ширина вывода для терминала.
Используется такими утилитами, как ls для форматирования вывода на терминал.
Если переменная не установлена, используется значение 80.

\textbf{LANGUAGE, LC\_ALL, LC\_TIME, LC\_CTYPE, TZ, LC\_NUMERIC и т.п.} - 
переменные окружения для локализации (задания часовых поясов и т.п).
Используется, например, командой ls для задания локали для вывода времени.

\subsection{Терминология}
\par\noindent\rule{\textwidth}{0.4pt}
\textbf{Команда} - выражение, составленное по правилам командного интерпретатора.
Может состоять из различных операторов или имён программ. Командный интерпретатор
занимается разбором таких выражений и их выполнением. Пример: VAR=1 VAR1=\$VAR
или /usr/bin/cat message > /dev/pts/27

\textbf{Микрокоманда} - элементарное действие из которого состоит команда.
Каждая микрокоманда выполняется в течение одного машинного такта.

\textbf{Нанокоманда} - каждой микрокоманде соответствует какая-то из нанокоманд.
Нанокоманды используются для формирования сигналов управления при выполнении микрокоманды.

\textbf{Утилита} - подмножество программ, предназначенных для работы в окружении
операционной системы или её администрирования. Утилиты не выполняют прикладных
задач, а лишь обеспечивают возможность работы пользователя или других программ
в системе. Пример: /usr/bin/login или /usr/bin/chmod.

\textbf{Программа} - исполняемый код или последовательность инструкций. Обычно
программа представлена в виде исполняемого файла или сценария на языке командного
интепретатора.

\subsection{Вывод}
\par\noindent\rule{\textwidth}{0.4pt}
В ходе выполнения этой работы я осознал и исследовал некоторые команды.
Стало ясно - команды в терминале это круто. С их помощью можно организовывать
работу многопользовательских систем, работу с данными и т.п.
\end{document}
