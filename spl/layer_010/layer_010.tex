\documentclass[12pt, a4paper]{article}
\usepackage[a4paper, includeheadfoot, mag=1000, left=2cm, right=1.5cm, top=1.5cm, bottom=1.5cm, headsep=0.8cm, footskip=0.8cm]{geometry}
% Fonts
\usepackage{fontspec, unicode-math}
\setmainfont[Ligatures=TeX]{CMU Serif}
\setmonofont{CMU Typewriter Text}
\usepackage[english, russian]{babel}
% Indent first paragraph
\usepackage{indentfirst}
\setlength{\parskip}{5pt}
% Diagrams
\usepackage{graphicx}
\usepackage{float}
% Source code
\usepackage{verbatim}
% Page headings
\usepackage{fancyhdr}
\pagestyle{fancy}
\renewcommand{\headrulewidth}{0pt}
\setlength{\headheight}{16pt}
%\newfontfamily\namefont[Scale=1.2]{Gloria Hallelujah}
\fancyhead{}
\begin{document}

% Title page
\begin{titlepage}
\begin{center}

\textsc{ФГАОУ ВО «Санкт-Петербургский национальный исследовательский университет информационных технологий, механики и оптики»\\[4mm]
Кафедра вычислительной техники}
\vfill
\textbf{Лабораторная работа №2\\[4mm]
Системное программное обеспечение\\[4mm]
}
вариант: \textit{chgrp} \\[16mm]
Саржевский Иван Анатольевич
\\[2mm]Группа P3202
\vfill
Санкт-Петербург\\[2mm]
2019 г.

\end{center}
\end{titlepage}


\section{О chgrp}
\textbf{\texttt{chgrp}} -- (eng. \textit{change group}) -- поменять
    группу пользователей, владеющих файлом. \texttt{chrp [OPTION]... GROUP FILE...}
\begin{itemize}
    \item \textbf{\texttt{-v, --verbose}} -- показывать информационное сообщение
      о каждом обработанном файле
    \item \textbf{\texttt{-c, --changes}} -- показывать информационное сообщение
      только в случае успешной обработки
    \item \textbf{\texttt{-f, --silent, --quiet}} -- подавлять большинство
      сообщений об ошибках
    \item \textbf{\texttt{--reference}} -- выставить группу
      согласно другому файлу
    \item \textbf{\texttt{-R, --recursive}} -- модифицировать файлы и директории
      рекурсивно
\end{itemize}

\section{nroff и troff}
Системы форматирования текстов, которые включают в себя язык разметки. Получают
на вход текст, содержащий макросы и директивы и преобразуют его.

nroff является неотъемлемой частью Unix-подобных систем, так как система справок
(man) опирается на него. troff добавляет дополнительную функциональность языку
разметки nroff, при этом nroff в тексте будет игнорировать директивы, которые
он не способен исполнить.

\section{Текст руководства}
{\scriptsize \verbatiminput{yasuo-chgrp.1}}

\section{Описание директив}
\begin{itemize}
  \item \textbf{\texttt{.TH}} -- заголовок man-страницы.
  \item \textbf{\texttt{.SH}} -- новый раздел
  \item \textbf{\texttt{.SS}} -- новый подраздел
  \item \textbf{\texttt{.PP}} -- новый параграф
  \item \textbf{\texttt{.TP}} -- TermParagraph
  \item \textbf{\texttt{.B}} -- жирный текст
  \item \textbf{\texttt{.I}} -- курсив
\end{itemize}

\section{Изменения стартового командного файла}
\begin{verbatim}
export MANPATH=/home/keker/code/yasuo/layer_010/man:$MANPATH
\end{verbatim}

\section{Результат}
{\scriptsize \verbatiminput{file}}

\section{Вывод}
В ходе выполнения данной лабораторной работы я познакомился с системой
форматирования текстов \texttt{nroff}, составил и описал на языке \texttt{roff}
страницу для \texttt{man} и зарегистрировал её.

\end{document}
