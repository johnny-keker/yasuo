\documentclass[12pt, a4paper]{article}
\usepackage[a4paper, includeheadfoot, mag=1000, left=2cm, right=1.5cm, top=1.5cm, bottom=1.5cm, headsep=0.8cm, footskip=0.8cm]{geometry}
% Fonts
\usepackage{fontspec, unicode-math}
\setmainfont[Ligatures=TeX]{CMU Serif}
\setmonofont{CMU Typewriter Text}
\usepackage[english, russian]{babel}
% Indent first paragraph
\usepackage{indentfirst}
\setlength{\parskip}{5pt}
% Diagrams
\usepackage{graphicx}
\usepackage{float}
% Source code
\usepackage{verbatim}
% Page headings
\usepackage{fancyhdr}
\pagestyle{fancy}
\renewcommand{\headrulewidth}{0pt}
\setlength{\headheight}{16pt}
%\newfontfamily\namefont[Scale=1.2]{Gloria Hallelujah}
\fancyhead{}
\begin{document}

% Title page
\begin{titlepage}
\begin{center}

\textsc{ФГАОУ ВО «Санкт-Петербургский национальный исследовательский университет информационных технологий, механики и оптики»\\[4mm]
Кафедра вычислительной техники}
\vfill
\textbf{Самостоятельная работа №1\\[4mm]
Системное программное обеспечение\\[4mm]
Знакомство с редактором vim\\[16mm]
}
\begin{flushright}
Саржевский Иван Анатольевич
\\[2mm]Группа P3202
\end{flushright}
\vfill
Санкт-Петербург\\[2mm]
2019 г.

\end{center}
\end{titlepage}

\section{Режимы}
\begin{itemize}
  \item \texttt{NORMAL} - интерпретация ввода как команд навигации и работы с
    текстом
  \item \texttt{INSERT} - набор текста
  \item \texttt{COMMAND LINE} - управление редактором
  \item \texttt{VISUAL} - режим выделения текста
\end{itemize}
В режим \texttt{INSERT} из \texttt{NORMAL} можно перейти несколькими командами,
в зависимости от положения начала ввода:
\begin{itemize}
  \item \texttt{i} - перед курсором
  \item \texttt{a} - после курсора
  \item \texttt{I} - в начале строки
  \item \texttt{A} - в конце строки
  \item \texttt{o} - на новой строке, после строки с курсором
  \item \texttt{O} - на новой строке, до строки с курсором
\end{itemize}
Из \texttt{INSERT} в \texttt{NORMAL} переходят по нажатию клавиши \texttt{ESC},
по нажатию \texttt{:} в режиме \texttt{NORMAL} происходит переход в режим
\texttt{COMMAND LINE}, по нажатию \texttt{Ctrl-V} - в режим \texttt{VISUAL}

\section{Работа с файлами}
В режиме \texttt{COMMAND LINE}:
\begin{itemize}
  \item \texttt{w} - записать изменения в файл
  \item \texttt{w <file>} - создать новый файл и записать в него
  \item \texttt{w! <file>} - аналогично, но с разрешением на перезапись
  \item \texttt{e} - считать файл с диска
  \item \texttt{e <file>} - открыть файл для редактирования
  \item \texttt{e!} - считать файл, не сохраняя изменения текущего
\end{itemize}

\section{Навигация}
В режиме \texttt{NORMAL}:
\begin{itemize}
  \item \texttt{h, l, j, k} - курсор влево, вправо, вниз и вверх
  \item \texttt{w} - к началу следующего слова
  \item \texttt{b} - к началу предыдущего слова
  \item \texttt{e} - к концу слова
  \item \texttt{\$} - к концу строки
  \item \texttt{0} - к началу строки
  \item \texttt{\{} - к предыдущему параграфу
  \item \texttt{\}} - к следующему параграфу
  \item \texttt{gg} - к началу файла
  \item \texttt{G} - к концу файла
  \item \texttt{n G} - к строчке \texttt{n}
\end{itemize}

\section{Работа с текстом}
В режиме \texttt{NORMAL}:
\begin{itemize}
\item \texttt{yy}: скопировать строку
\item \texttt{dd}: вырезать строку
\item \texttt{p}: вставить после курсора
\item \texttt{P}: вставить перед курсором
\item \texttt{J}: присоединить следующую строку к текущей
\item \texttt{u}: отменить последнее изменение
\end{itemize}

\section{Поиск и замена}
В режиме \texttt{COMMAND LINE} команда \texttt{/} инициирует поиск по тексту.
\begin{itemize}
  \item \texttt{:s/search/replace/}: заменить первое вхождение в текущей строке;
  \item \texttt{:s/search/replace/g}: заменить все вхождения в текущей строке;
  \item \texttt{:\%s/search/replace/}: заменить первое вхождение в каждой строке;
  \item \texttt{:\%s/search/replace/g}: заменить все вхождения в тексте.
\end{itemize}

\section{Несколько файлов}
В режиме \texttt{NORMAL}:
\begin{itemize}
  \item \texttt{e} - открыть файл
  \item \texttt{ls} - отобразить открытые файлы
  \item \texttt{b <file>} - переход к редактированию другого файла
  \item \texttt{sp} - новое горизонтальное окно
  \item \texttt{vs} - новое вертикальное окно
  \item \texttt{Ctrl-w} - переключение между открытыми окнами
  \item \texttt{tabnew} - новая вкладка
  \item \texttt{tabe <file>} - открыть файл в новой вкладке
  \item \texttt{tabnext, tabprevious} - навигация по вкладкам
\end{itemize}

\section{Параметры редактора}
Файл \texttt{.vimrc} из домашней директории исполняется при каждом открытии
редактора, в нем можно его конфигурировать, например выставлять определенные
опции с помощью команды \texttt{set}.

\texttt{set mouse=a} позволит включить навигацию по клику мыши во всех режимах
 
\section{Выход}
В режиме \texttt{NORMAL}:
\begin{itemize}
  \item \texttt{:q} - выйти, если нет несохраненных изменений;
  \item \texttt{:q!} - выйти без сохранения изменений;
  \item \texttt{:wq} - записать файл и выйти;
  \item \texttt{:x} - записать файл, если есть несохраненные изменения, и выйти.
\end{itemize}

\end{document}
